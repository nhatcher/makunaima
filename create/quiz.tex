%BEGINFIELDS
nombre(7)=Nombre: 
carnet(4)=Carnet:
seccion(2)=Grupo:
professor(7)=Professor: 
%ENDFIELDS

%BEGINVARS
universidad=Universidad Simon Bolivar
departamento=\begin{center}Departamento de Matem\'aticas\\Puras y Aplicadas\end{center}
trimestre=Enero-Marzo 2010
titulo=$4^{\mbox{to}}$ Ex\'amen Parcial (20\%)
materia=Matem\'aticas II (MA-1112)
logo=\includegraphics[scale=0.2]{usb}
%ENDVARS

%BEGINSTYLE
ExamStylePDF=JeanPierre
HeadStylePDF=JeanPierre
ExamStyleHTMP=Plain
HeadStylePDF=Plain
%ENDSTYLE

%BEGINDEFS
\documentclass[12pt]{article}
\usepackage[utf8]{inputenc}                                                                               
\usepackage[spanish]{babel}    
\usepackage{amsmath,amstext,amscd,amsfonts,amssymb,amsthm}
\usepackage{epsfig}         
\newcommand{\re}{\mathbb R}                                                                               
\newcommand{\Ln}{\textrm{Ln}}                                                                             
%ENDDEFS

%BEGINEXAM                                                                                         
\begin{Pregunta}{3}{1}                                                                                    
El valor del ${\displaystyle\int_{-\infty}^\infty\frac{{\rm d}x}{1+x^2}}$ es igual a:                  
\Opcion                                                                                                   
${\displaystyle\frac{\pi}{2}}$                                                                            
\Opcion                                                                                                   
${\displaystyle\pi}$                                                                                      
\Opcion                                                                                                   
No converge                                                                                               
\Opcion                                                                                                   
${\displaystyle-\frac{\pi}{4}}$                                                                           
\end{Pregunta}                                                                                            


\begin{Pregunta}{3}{3}
La integral ${\displaystyle\int\frac{{\rm d}x}{1-{\rm sen}x+{\rm cos}x}}$ es igual a:
\Opcion                                                                              
${\displaystyle{\rm ln}\left|1+{\rm tg}\frac{x}{2}\right|+C}$                        
\Opcion                                                                              
${\displaystyle-{\rm ln}\left|1+{\rm tg}\frac{x}{2}\right|+C}$                       
\Opcion                                                                              
${\displaystyle{\rm ln}\left|1-{\rm tg}\frac{x}{2}\right|+C}$                        
\Opcion                                                                              
${\displaystyle-{\rm ln}\left|1-{\rm tg}\frac{x}{2}\right|+C}$                       
\end{Pregunta}

\begin{Pregunta}{4}{2}
El vol\'umen del s\'olido generado al rotar alrededor del eje $Y$ la regi\'on acotada por el eje $Y$ y las gr\'aficas de las funciones ${\displaystyle y=x^3}$, ${\displaystyle y=1}$ y ${\displaystyle y=8}$ es:
\Opcion
${\displaystyle 83\pi}$
\Opcion
${\displaystyle\frac{93}{15}\pi}$
\Opcion
${\displaystyle\frac{93}{5}\pi}$
\Opcion
${\displaystyle\frac{8}{5}\pi}$
\end{Pregunta}


\begin{Pregunta}{3}{0}
El \'area de la regi\'on encerrada por las gr\'aficas de las funciones ${\displaystyle f(x)=2x-x^2}$ y ${\displaystyle g(x)=x-2}$ es igual a:
\Opcion
${\displaystyle\frac{9}{2}}$
\Opcion
${\displaystyle-\frac{9}{2}}$
\Opcion
${\displaystyle\frac{9}{4}}$
\Opcion
${\displaystyle-\frac{3}{4}}$
\end{Pregunta}

\begin{Pregunta}{4}{0}
La integral ${\displaystyle\int\frac{2x^2-5x+2}{x^3+x}{\rm d}x}$ es igual a:
\Opcion
${\displaystyle2{\rm ln}|x|-5{\rm arctg}(x)+C}$
\Opcion
${\displaystyle2{\rm ln}|x|-\frac{1}{\sqrt{5}}{\rm arctg}(\frac{x}{\sqrt{5}})+C}$
\Opcion
${\displaystyle{\rm ln}\left|\frac{-x^2-5x+1}{x^3+x}\right|+C}$
\Opcion
${\displaystyle2{\rm ln}\left|\frac{1-x^2}{x^3+x}\right|+C}$
\end{Pregunta}

\begin{Pregunta}{3}{0}
La integral ${\displaystyle\int x^3e^{x^2}{\rm d}x}$ es igual a:
\Opcion
${\displaystyle\frac{1}{2}e^{x^2}\left(x^2-1\right)+C}$
\Opcion
${\displaystyle e^{x^2}\left(x^2-1\right)+C}$
\Opcion
${\displaystyle\frac{1}{4}e^{x^2}\left(1-x^2\right)+C}$
\Opcion
${\displaystyle-e^{x^2}\left(x^2-1\right)}$
\end{Pregunta}
%ENDEXAM